% resume.tex
%
% (c) 2002 Matthew Boedicker <mboedick@mboedick.org> (original author) http://mboedick.org
% (c) 2003-2007 David J. Grant <davidgrant-at-gmail.com> http://www.davidgrant.ca
%
% This work is licensed under the Creative Commons Attribution-ShareAlike 3.0 Unported License. To view a copy of this license, visit http://creativecommons.org/licenses/by-sa/3.0/ or send a letter to Creative Commons, 171 Second Street, Suite 300, San Francisco, California, 94105, USA.

\documentclass[letterpaper,11pt]{article}

%-----------------------------------------------------------
%Margin setup

\setlength{\voffset}{0.1in}
\setlength{\paperwidth}{8.5in}
\setlength{\paperheight}{11in}
\setlength{\headheight}{0in}
\setlength{\headsep}{0in}
\setlength{\textheight}{11in}
\setlength{\textheight}{9.5in}
\setlength{\topmargin}{-0.25in}
\setlength{\textwidth}{7in}
\setlength{\topskip}{0in}
\setlength{\oddsidemargin}{-0.25in}
\setlength{\evensidemargin}{-0.25in}
%-----------------------------------------------------------
%\usepackage{fullpage}
\usepackage{shading}
%\textheight=9.0in
\pagestyle{empty}
\raggedbottom
\raggedright
\setlength{\tabcolsep}{0in}

%-----------------------------------------------------------
%Custom commands
\newcommand{\resitem}[1]{\item #1 \vspace{-2pt}}
\newcommand{\resheading}[1]{{\large \parashade[.9]{sharpcorners}{\textbf{#1 \vphantom{p\^{E}}}}}}
\newcommand{\ressubheading}[4]{
\begin{tabular*}{6.5in}{l@{\extracolsep{\fill}}r}
		\textbf{#1} & #2 \\
		\textit{#3} & \textit{#4} \\
\end{tabular*}\vspace{-6pt}}
%-----------------------------------------------------------


\begin{document}

\begin{tabular*}{7in}{l@{\extracolsep{\fill}}r}
\textbf{\Large Guruprerana Shabadi}  & \\
103, avenue Henri Becquerel & guruprerana.github.io \\
Palaiseau, 91120, France & guruprerana.shabadi@polytechnique.edu \\
\end{tabular*}
\\

\vspace{0.1in}

\resheading{Éducation}
\begin{itemize}
\item
	\ressubheading{École Polytechnique}{Palaiseau, France}{Double license: Mathematiques et Informatique (GPA: 4.2/4.0)}{Septembre 2019 - Present}
	\begin{itemize}
		\resitem{Admis avec distinction}
	\end{itemize}

\item
	\ressubheading{Sindhi High School}{Bengaluru, India}{Diplôme d'études secondaires  (Note finale: 95.6\%)}{Juin 2017 - Avril 2019}
	\begin{itemize}
		\resitem{Mathematiques, Physique, Chimie, et Biologie}
	\end{itemize}

\end{itemize}

\resheading{Experience}
\begin{itemize}
\item
	\ressubheading{Anemone}{}{Projet entrepreneurial}{Oct. 2020 ---}
	\begin{itemize}
		\resitem{Direction de l'équipe de développement chez Anemone.}
		\resitem{Construction d'un réseau socioprofessionnel adapté aux besoins de l'ensemble des industries culturelles.}
	\end{itemize}
\item
	\ressubheading{INRIA}{Paris, France}{Thèse de L3}{Janvier 2022 - Avril 2022}
	\begin{itemize}
		\resitem{Supervisé par Caterina Urban (Équipe ANTIQUE)}
		\resitem{Sujet: Analyse statique des logiciels de science des données}
	\end{itemize}
\item
	\ressubheading{Department Informatique, École Polytechnique (DIX)}{Palaiseau, France}{Stage de recherche}{Juin 2021 - Decembre 2021}
	\begin{itemize}
		\resitem{Supervisé par Sergio Mover (Équipe Cosynus)}
		\resitem{Sujet: Étude des délais de communication dans les systèmes cyber-physiques}
	\end{itemize}

\item 
	\ressubheading{Fasfox}{Paris, France}{Stagiaire développeur full-stack}{Juin 2020 - Août 2020}
	\begin{itemize}
		\resitem{Travail sur le développement d'une application appelée \textit{Concrete Dispatch} utilisée par les grands chantiers de construction. Impliqué dans tous les aspects du développement, y compris le frontend et le backend.}
		\resitem{Logiciels utilisés: Django, Vue.js, Android}
	\end{itemize}
\end{itemize}

\resheading{Awards}
	\begin{tabular*}{6.5in}{l@{\extracolsep{\fill}}r}
		Bourse d'excellence de la Fondation de l'École Polytechnique & 2019-2022\\
		Bourse de la Direction Générale de l'Armement (DGA) & 2021-2022\\
\end{tabular*}

\resheading{Projets}
\begin{itemize}
\item
	\ressubheading{Nuit de l'IA Hackathon}{}{Organisée par Binet AI et l'accélérateur de start-ups X-UP}{Septembre 2019}
	\begin{itemize}
		\resitem{Membre de l'équipe gagnante du hackathon}
		\resitem{Meilleure implémentation d'un algorithme de classification multi-label pour les produits de supermarché}
	\end{itemize}

\end{itemize}

\resheading{Compétences}

\begin{description}
\item[Langages de programmation:]
C/C++, Python, Coq, Haskell, JavaScript, Java, HTML-CSS, x86 Assembly
\item[Frameworks:]
Django, Vue.js, Android, React, React Native, Pandas, Scikit-learn, Tensorflow, Qt
\item[Langues:]
Français (C1), Anglais (natif), Kannada (natif), Hindi
\end{description}

\resheading{Expérience de service volontaire}
\begin{itemize}
\item
	\ressubheading{Developer Student Club (DSC) de l'École Polytechnique}{}{Responsable du DSC}{Septembre 2020 - Juin 2021}
	\begin{itemize}
		\resitem{Dans le cadre du programme DSC de Google, j'ai dirigé une équipe chargée d'organiser des événements et des concours sur diverses plateformes et technologies, notamment le cloud et l'IA.}
	\end{itemize}

\item 
	\ressubheading{BDE du Bachelor de l'École Polytechnique}{}{Directeur Financier}{Juin 2020 - Juin 2021}
	\begin{itemize}
		\resitem{Responsable de la comptabilité et de la collecte de parrainages et de fonds, ainsi que de l'organisation de la vie sociale et culturelle sur le campus.}
	\end{itemize}
\end{itemize}

\end{document}
